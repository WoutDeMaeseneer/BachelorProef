%%=============================================================================
%% Conclusie
%%=============================================================================

\chapter{Conclusie}%
\label{ch:conclusie}

% TODO: 
% Trek een duidelijke conclusie, in de vorm van een antwoord op de onderzoeksvra(a)g(en). V
% Wat was jouw bijdrage aan het onderzoeksdomein en V
% hoe biedt dit meerwaarde aan het vakgebied/doelgroep? V
% Reflecteer kritisch over het resultaat. In Engelse teksten wordt deze sectie
% ``Discussion'' genoemd. Had je deze uitkomst verwacht? Zijn er zaken die nog
% niet duidelijk zijn?
% Heeft het onderzoek geleid tot nieuwe vragen die uitnodigen tot verder 
%onderzoek?

Uit dit onderzoek kan er eerst en vooral geconcludeerd worden dat data van een smart card met een card reader naar een webapplicatie versturen niet iets voor de hand liggend is. Er is gebleken dat WebUSB, dat voor het getest werd een simpele oplossing leek te zijn, geen oplossing is. Dit komt omdat de card reader niet kan worden gebruikt met WebUSB wegens security issues, omdat er mogelijk belangrijke data aan te pas komt bij het inlezen van de smart card. Hierdoor voldoet het niet aan de succesfactor dat het moet werken en is het geen oplossing tot deze onderzoeksvraag.
Er werd ook een interview afgelegd met Anton D'hondt waarin hij zijn idee tot een mogelijke oplossing uitlegde. Zijn idee werd vervolgens uitgewerkt tot een Proof of Concept. En uit deze Proof of Concept kan geconcludeerd worden dat het een oplossing is omdat het werkt en onder een seconde de card tag op de webapp laat verschijnen. Omdat het dus voldoet aan beide succesfactoren kan er geconcludeerd worden dat het dus een antwoord biedt op de onderzoeksvraag.
Het antwoord op de onderzoeksvraag biedt een meerwaarde voor Jan De Nul. Jan De Nul Human Resource wil de applicatie om MiFare-kaarten in te lezen omvormen tot een webapplicatie. Deze webapplicatie moet deel uitmaken van een website die de verzameling vormt van nog verschillende andere webapps. Om deze nieuwe webapplicatie te kunnen gebruiken moest het mogelijk zijn om met hun ACR122U card reader de data van een kaart uit te lezen naar de webapp. De Proof of Concept die ik voor deze bachelorproef heb uitgewerkt zal dus worden gebruikt in die webapplicatie.
De uitkomst dat uit deze bachelorproef is gekomen was niet wat werd verwacht. Voor het onderzoek leek het dat WebUSB gewoon een oplossing zou zijn en dat er zelfs nog meerdere oplossingen bestonden die door browsers worden ondersteund. Echter bleek dus dat de oplossing verder moest worden gezocht dan WebUSB. Ik denk wel dat de Proof of Concept duidelijk en goed in elkaar zit en het een volwaardige oplossing is tot de onderzoeksvraag.

