%%=============================================================================
%% Voorwoord
%%=============================================================================

\chapter*{\IfLanguageName{dutch}{Woord vooraf}{Preface}}%
\label{ch:voorwoord}

%% TODO:
%% Het voorwoord is het enige deel van de bachelorproef waar je vanuit je
%% eigen standpunt (``ik-vorm'') mag schrijven. Je kan hier bv. motiveren
%% waarom jij het onderwerp wil bespreken.
%% Vergeet ook niet te bedanken wie je geholpen/gesteund/... heeft
Beste lezer,

Met trots presenteer ik u mijn thesis, die het resultaat is van maandenlang onderzoek en hard werk. Het schrijven van deze thesis was een uitdagend en tegelijkertijd verrijkend proces, waarin ik veel heb geleerd en ontdekt.

Allereerst wil ik graag mijn oprechte dank uitspreken aan mijn co-promotor, Kevin Lommens, voor zijn waardevolle begeleiding, steun en feedback tijdens mijn onderzoeksproces. Zonder zijn begeleiding en inzichten had ik deze thesis niet kunnen voltooien.

Ook wil ik mijn dank uitspreken aan Anton D'hondt voor het deelnemen aan mijn onderzoek. Zijn waardevolle inbreng en medewerking was essentieel voor het verzamelen van de benodigde gegevens en het verkrijgen van inzichten. Ik heb nog nooit iemand ontmoet die zoveel weet van alles wat met programmeren te maken heeft.

Ik heb tijdens het werken aan deze bachelorproef dus ook heel veel kunnen leren van zowel Kevin als Anton. 

Als laatste wil ik ook nog Jan De Nul bedanken, meer bepaald de HR-afdeling. Ik zou nooit tot dit onderwerp gekomen zijn moest ik er geen stage mogen gaan doen. Ik vind persoonlijk dat dit onderwerp zeer interessant was om onderzoek rond te doen. Ook het feit dat het resultaat van deze bachelorproef ook echt iets bijbrengt vond ik zeer motiverend. En ik ben heel vereerd dat ze de Proof of Concept van deze bachelorproef willen gebruiken in een van hun webapplicaties.

Voor deze bachelorproef heb ik uitgezocht op welke manieren een kaart lezer kan worden gebruikt voor het uitlezen van smartcard data naar de webapplicatie. Hiervan heb ik ook een Proof of Concept gemaakt. De Proof of Concept wordt gebruikt in een webapplicatie van Jan De Nul, deze webapplicatie heb ik kunnen maken tijdens mijn stage bij Jan De Nul. Het was een zeer interessant en leuk onderwerp om uit te werken.