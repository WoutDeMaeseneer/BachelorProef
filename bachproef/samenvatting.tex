%%=============================================================================
%% Samenvatting
%%=============================================================================

% TODO: De "abstract" of samenvatting is een kernachtige (~ 1 blz. voor een
% thesis) synthese van het document.
%
% Een goede abstract biedt een kernachtig antwoord op volgende vragen:
%
% 1. Waarover gaat de bachelorproef?
% 2. Waarom heb je er over geschreven?
% 3. Hoe heb je het onderzoek uitgevoerd?
% 4. Wat waren de resultaten? Wat blijkt uit je onderzoek?
% 5. Wat betekenen je resultaten? Wat is de relevantie voor het werkveld?
%
% Daarom bestaat een abstract uit volgende componenten:
%
% - inleiding + kaderen thema V 
% - probleemstelling V
% - (centrale) onderzoeksvraag V
% - onderzoeksdoelstelling V
% - methodologie V
% - resultaten (beperk tot de belangrijkste, relevant voor de onderzoeksvraag) V
% - conclusies, aanbevelingen, beperkingen
%
% LET OP! Een samenvatting is GEEN voorwoord!

%%---------- Nederlandse samenvatting -----------------------------------------
%
% TODO: Als je je bachelorproef in het Engels schrijft, moet je eerst een
% Nederlandse samenvatting invoegen. Haal daarvoor onderstaande code uit
% commentaar.
% Wie zijn bachelorproef in het Nederlands schrijft, kan dit negeren, de inhoud
% wordt niet in het document ingevoegd.

\IfLanguageName{english}{%
\selectlanguage{dutch}
\chapter*{Samenvatting}
\selectlanguage{english}
}{}

%%---------- Samenvatting -----------------------------------------------------
% De samenvatting in de hoofdtaal van het document

\chapter*{\IfLanguageName{dutch}{Samenvatting}{Abstract}}

In deze bachelorproef zal er onderzocht worden op welke manier het beste een smartcard kan worden uitgelezen met een card reader waarvan de data van de smartcard als gevolg meteen op de webapplicatie kan gebruikt worden. Het thema voor deze bachelorproef zal dus draaien rond smartcards, card readers en de technologieën zoals Windows Services, die er aan te pas komen. Bij Jan De Nul willen ze hun applicatie voor het beheren van badges omvormen naar een webapplicatie. Deze webapplicatie zal deel uitmaken van een centrale webapplicatie die een verzameling vormt van nog andere webapplicaties. Een van de functies van deze webapp zal het registreren van badges in het systeem zijn. Hiervoor moet de badge eerst worden uitgelezen, waarbij de tag moet worden opgehaald. Het is de bedoeling dat er dus een smartcard tegen de lezer wordt gehouden en dat de tag van die kaart vrijwel meteen op de webapp verschijnt. Om dit mogelijk te maken wordt er in deze bachelorproef een oplossing gezocht. Vervolgens kan de kaart worden aangemaakt door nog data zoals type of naam eraan toe te voegen.
Voor het inlezen van de tag zodat die meteen in de webapp verschijnt was er echter nog geen geschikte oplossing gevonden. In deze bachelorproef wordt er dus naar een oplossing gezocht voor dit probleem. De onderzoeksvraag zal dus als volgt zijn: Wat is de beste manier om een card reader te verbinden met een webapplicatie? Door deze vraag te beantwoorden zal de best mogelijke manier gevonden worden om smartcard data te transporteren naar de webapp. Met als gevolg een werkende Proof of Concept die in gebruik kan worden genomen voor de Jan De Nul webapplicatie.
Hoe is er te werk gegaan? Eerst is erop zoek gegaan op het internet over welke potentiële oplossingen er al bestaan om dit probleem aan te pakken. Hieruit bleek WebUSB de enige 'directe' manier. WebUSB zit namelijk ingebouwd in de meeste browsers en zorgt ervoor dat websites met de toestemming van de gebruiker toegang kunnen krijgen om aangesloten USB-apparaten te gebruiken. WebUSB als enige oplossing is natuurlijk niet voldoende om een onderzoek rond te doen. Daarom is er de lead developer van de Jan De Nul HR-afdeling, Anton D'hondt gevraagd om een interview af te leggen. Hij had het idee om het probleem op te lossen met behulp van een Windows Service en het SignalR framework. Hierbij zal de code om de kaart uit te lezen in de Windows Service zitten en zal de service er ook voor zorgen dat de code wordt verstuurd naar de SignalR Hub. De SignalR Hub zal er vervolgens voor zorgen dat de data wordt doorgegeven naar de webapplicatie die ook een connectie heeft met de SignalR Hub. Nu de mogelijke oplossingen bekend zijn werden ze beide uitgetest. 
Uit de resultaten bleek dat WebUSB geen mogelijke oplossing is omdat het onmogelijk is om een card reader te connecteren wegens beveiligingsproblemen, omdat er mogelijks gevoelige data op smartcards kan staan. De oplossing dat bleek te werken is het idee van Anton D'hondt. Zijn idee is uitgewerkt tot een Proof of Concept dat wordt gebruikt door Jan De Nul. Deze Proof of Concept is dus een werkend resultaat dat als antwoord dient voor de onderzoeksvraag van deze bachelorproef.
