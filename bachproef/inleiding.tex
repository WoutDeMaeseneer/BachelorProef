%%=============================================================================
%% Inleiding
%%=============================================================================

\chapter{\IfLanguageName{dutch}{Inleiding}{Introduction}}%
\label{ch:inleiding}

Wilt u op uw webapplicatie smartcard date kunnen inlezen met behulp van een cardreader? In deze bachelorproef gaat worden uitgezocht wat de beste manier is om dit te doen.


%De inleiding moet de lezer net genoeg informatie verschaffen om het onderwerp te begrijpen en in te zien waarom de onderzoeksvraag de % waard is om te onderzoeken. In de inleiding ga je literatuurverwijzingen beperken, zodat de tekst vlot leesbaar blijft. Je kan de %inleiding verder onderverdelen in secties als dit de tekst verduidelijkt. Zaken die aan bod kunnen komen in de %inleiding~\autocite{Pollefliet2011}:

%\begin{itemize}
%  \item context, achtergrond
%  \item afbakenen van het onderwerp
%  \item verantwoording van het onderwerp, methodologie
%  \item probleemstelling
%  \item onderzoeksdoelstelling
%  \item onderzoeksvraag
%  \item \ldots
%{itemize}

\section{\IfLanguageName{dutch}{Probleemstelling}{Problem Statement}}%
\label{sec:probleemstelling}

De doelgroep voor deze bachelorproef is Jan De Nul, meer bepaald de HR-afdeling. De HR-afdeling beheert alle badges binnen het bedrijf. Hier hebben ze ook een applicatie voor. Maar hiervoor willen ze een nieuwe webapplicatie omdat de oude verouderd is. Deze webapplicatie zal deel uitmaken van een verzameling van andere webapplicaties die de HR-afdeling gebruikt. Een van de functionaliteiten van de webapp is het aanmaken van badges in het systeem. Hiervoor moet de badge worden gescand en wordt er verwacht dat de badge tag vrijwel meteen in de website wordt weergegeven. Hiervoor hebben ze nog geen manier gevonden, maar daar zal het resultaat van deze bachelorproef verandering in moeten brengen.

\section{\IfLanguageName{dutch}{Onderzoeksvraag}{Research question}}%
\label{sec:onderzoeksvraag}
Wat is de beste manier om een driver te maken die een card reader rechtstreeks kan verbinden met een webapplicatie in de browser?

\section{\IfLanguageName{dutch}{Onderzoeksdoelstelling}{Research objective}}%
\label{sec:onderzoeksdoelstelling}
Het beoogde resultaat van deze bachelorproef zal de beste manier zijn vanaf de webapplicatie gebruik te maken van een card reader. Of het resultaat succesvol is zal bepaald worden aan de hand van een Proof of Concept. Deze Proof of Concept zal dus binnen de applicatie van Jan De Nul volledig moeten werken om als succesvol aanschouwd te worden.


\section{\IfLanguageName{dutch}{Opzet van deze bachelorproef}{Structure of this bachelor thesis}}%
\label{sec:opzet-bachelorproef}

% Het is gebruikelijk aan het einde van de inleiding een overzicht te
% geven van de opbouw van de rest van de tekst. Deze sectie bevat al een aanzet
% die je kan aanvullen/aanpassen in functie van je eigen tekst.

De rest van deze bachelorproef is als volgt opgebouwd:

In Hoofdstuk~\ref{ch:stand-van-zaken} wordt een overzicht gegeven van de stand van zaken binnen het onderzoeksdomein, op basis van een literatuurstudie.

In Hoofdstuk~\ref{ch:methodologie} wordt de methodologie toegelicht en worden de gebruikte onderzoekstechnieken besproken om een antwoord te kunnen formuleren op de onderzoeksvragen.

De methodologie zal beginnen met een korte inleiding waarin ook de structuur van de methodologie zal beschreven worden voor extra duidelijkheid. Dan zal er een korte samenvatting komen van de twee mogelijke oplossingen die gevonden zijn om aan te geven wat er zal onderzocht worden.
Vervolgens zal de eerste manier, met behulp van WebUSB omschreven worden. Dit zal het volgende inhouden: Een inleiding, een projectje waarin de manier met WebUSB werd getest, tenslotte waarom WebUSB uiteindelijk geen geschikte oplossing bleek te zijn in het geval van card readers.
Daarna zal het interview met Anton D'hondt worden overlopen. Dit zal een omschrijving zijn van het idee dat hij in gedachten had om dit probleem op te lossen.
In het laatste deel van de methodologie zal het idee van D'hondt worden uitgewerkt. Dit zal een inleiding bevatten alsook het idee opgesplitst in de volgende drie stappen: Windows Services, SignalR Hub en SignalR in de webapp.

In Hoofdstuk~\ref{ch:conclusie}, tenslotte, wordt de conclusie gegeven en een antwoord geformuleerd op de onderzoeksvragen. Daarbij wordt ook een aanzet gegeven voor toekomstig onderzoek binnen dit domein.