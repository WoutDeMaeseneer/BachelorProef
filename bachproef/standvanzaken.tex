\chapter{\IfLanguageName{dutch}{Stand van zaken}{State of the art}}%
\label{ch:stand-van-zaken}

% Tip: Begin elk hoofdstuk met een paragraaf inleiding die beschrijft hoe
% dit hoofdstuk past binnen het geheel van de bachelorproef. Geef in het
% bijzonder aan wat de link is met het vorige en volgende hoofdstuk.

% Pas na deze inleidende paragraaf komt de eerste sectiehoofding.

\section{Introductie}
Bij het verbinden van een card reader met de webapplicatie komen verschillende technologieën aan te pas. Om ervoor te zorgen dat u alles goed begrijpt in deze bachelorproef is deze stand van zaken opgesteld die u de nodige inleiding geeft tot de technologieën die voorbij zullen komen in deze bachelorproef.  

Hierin zal er info staan over card readers, hoe een card reader werkt, welke soorten card readers er zijn en meer info over de ACR122U card reader die zal worden gebruikt in de Proof of Concept. Ook zal er uitleg zijn over de contactloze smartcard, de karakterisering ervan, de technologie die erachter zit en de voor- en nadelen ervan met nog extra info over de MiFare-kaart die zal gebruikt worden voor de Proof of Concept.  

Vervolgens gaat er info zijn over hoe een smartcard kan uitgelezen worden, dit is belangrijk omdat het in de PoC de bedoeling zal zijn dat de tag kan worden opgehaald om die vervolgens richting de webapplicatie te versturen. Verder zal er informatie zijn over WebUSB, dat vanuit de front-end al voor de verbinding met USB-apparaten kan zorgen. Echter zullen Server-Sent Events en de SignalR library van groter belang zijn voor deze bachelor proef, deze zullen namelijk zorgen voor de real-time webfunctionaliteit. SignalR werkt met Hubs en Clients die voor de communicatie moeten zorgen, hierover zal meer uitleg volgen. 

Om het plaatje compleet te maken zal er nog info zijn over Windows Services, deze zullen op de achtergrond draaien en wanneer er een smartcard wordt gescand de tag van die smartcard doorsturen naar de webapplicatie met behulp van de eerder vermelde SignalR library.


\section{Card Readers}
\subsection{Inleiding tot de card reader}
``Een kaartlezer is in wezen een stukje elektronische technologie dat ontworpen is om de informatie in de magneetstrip of chip van de kaart te decoderen. Het elektronische apparaat voor gegevensinvoer dat de kaartlezer is, kan in de hand worden gehouden, draadloos zijn, of via een kabel zijn aangesloten op een pc of een verkoopterminal, en heeft verschillende gemeenschappelijke kenmerken voor de verschillende apparaten die op de markt verkrijgbaar zijn.''\autocite{MarinPenchevCardReader}

Een kaartlezer kan volgens \textcite{MarinPenchevCardReader} ook enkele functies bevatten die de kaartlezer de mogelijkheid geven om de kaart van een klant in te lezen. De functies kunnen bestaan uit het fysiek uitlezen van de kaart, de mogelijkheid te bieden om de kaart uit te lezen door de magneetstrip van de kaart tegen het apparaat te vegen, of de aanwezigheid van een pinpad of touchpad. Ook kan het bijvoorbeeld de functie hebben om contactloze transacties te verwerken. Kaartlezers kunnen tenslotte een scherm, toetsenbord of touchscreen bevatten die het inlezen van de kaart vergemakkelijken.

Volgens \textcite{MarinPenchevCardReader} zijn er verschillende types van card reader. 
\begin{itemize}
    \item De mobile card reader is zoals de naam zegt mobiel, ze zijn geconnecteerd over het internet en kunnen worden teruggevonden in bijvoorbeeld taxi's.
    \item De portable card reader is ook een mobiele soort card reader, het verschil tussen de mobile en de portable card readers is dat de mobile card reader gemaakt is om in beweging te gebruiken.
    \item Ook zijn er bijvoorbeeld de countertop card reader, deze is te vinden bij bijvoorbeeld de kassa van een winkelen.
    \item Dan zijn er ook nog de payment gerelateerde card readers zoals de magnetic stripe en de chip\&PIN.
    \item En ten slotte ook nog de contactloze smartcard reader. Aangezien er voor de Proof of Concept van deze bachelorproef een ACR122U smartcard reader gaat worden gebruikt, omdat Jan De Nul deze card reader reeds gebruikt, gaat er gefocust worden op smartcard readers.
\end{itemize}





\section{Smartcard Readers}
\subsection{Inleiding tot de smart card reader}
Volgens \textcite{OnyaitOdekeCardReader} zijn smartcard readers apparaten die kunnen worden gebruikt om een smartcard uit te lezen. Enkele voorbeelden hiervan zijn kredietkaarten, debet kaarten en medische verzekeringskaarten. Een smartcard is een plastiek kaart die een geïnstalleerd gecoördineerd circuit bevat dat kan bestaan uit enerzijds een microcontroller of anderzijds een geheugenchip. Op een smartcard kan daardoor heel wat informatie worden opgeslagen, gecodeerd en gevalideerd.

Wat betreft het communicatie protocol maakt de ACR122U-A9 smart card reader (die voor de PoC gaat gebruikt worden) gebruik van het contactless protocol, via de contactless interface ISO/IEC 14443. Als een kaart geen gebruik maakt van een standard transmission protocol, maar wel een custom/proprietary protocol, dan wordt het communicatie protocol aangeduid als T=14.

Readers die tot de 0x0B klasse behoren hebben geen drivers nodig als ze gebruikt worden via PC/SC-compliant operating systems, dit is omdat het besturingssysteem al standaard de driver levert. Dit is mogelijk omdat de laatste PC/SC CCID specificaties een nieuw smartcard framework definiëren dat werkt met USB devices die behoren tot de 0x0B device class.


``Elke lezer moet worden gedefinieerd voor gebruik door het smartcard-subsysteem. Het subsysteem is niet verantwoordelijk voor een lezer die er niet specifiek aan is gegeven.``\autocite{MicrosoftSmartCardReaders}

Elke smartcard lezer kan volgens \textcite{MicrosoftSmartCardReaders} dus onder een bepaalde lezersgroep vallen, dit kunnen echter ook meerdere lezersgroepen zijn. Deze groepen worden bepaald door het subsysteem, gebruikers en beheerders.

\subsection{Voordelen van een smart card reader}
Volgens \textcite{OnyaitOdekeCardReader} zijn de twee grote voordelen van een smartcard reader dat het veilig is en dat het veel wordt gebruikt. Wat beveiliging betreft maken ze gebruik van encryptie en validatie technologie. Toegang tot de informatie op de kaart kan enkel worden verworven als de chip in de kaart contact maakt met de card reader. Smartcards worden ook heel veel gebruikt, denk maar aan het betalen in de supermarkt, het gebruik als toegangsbadge of het scannen van de kaart bij het opstappen in het openbaar vervoer.

\subsection{ACR122U card reader}
De card reader die bij Jan De Nul wordt gebruikt en dus ook zal gebruikt worden voor deze Proof of Concept is de ACR122U-A9. Lees de info over deze reader gerust eens door indien u meer info wilt over bijvoorbeeld de protocollen dat deze reader gebruikt of andere specificaties hiervan.

``De ACR122U NFC Reader is een PC-gekoppelde contactloze smartcard lezer/schrijver ontwikkeld op basis van 13,56 MHz Contactloze (RFID) Technologie. Hij voldoet aan de ISO/IEC18092 standaard voor Near Field Communication (NFC) en ondersteunt niet alleen MIFARE® en ISO 14443 A en B kaarten, maar ook alle vier typen NFC tags.
ACR122U is compatibel met zowel CCID als PC/SC. Het is dus een plug-and-play USB-apparaat dat interoperabiliteit met verschillende apparaten en toepassingen mogelijk maakt. Met een toegangssnelheid tot 424 kbps en een volledige USB-snelheid tot 12 Mbps kan ACR122U ook sneller en efficiënter lezen en schrijven. De nabijheidsafstand van de ACR122U is maximaal 5 cm, afhankelijk van het type contactloze tag dat wordt gebruikt.
Om het beveiligingsniveau te verhogen kan de ACR122U worden geïntegreerd met een ISO 7816-3 SAM-slot. Bovendien is de ACR122U NFC-lezer beschikbaar in modulevorm, zodat hij gemakkelijk kan worden geïntegreerd in grotere machines, zoals POS-terminals, fysieke toegangssystemen en verkoopautomaten.
De ACR122U NFC Reader is ideaal voor zowel veilige persoonlijke identiteitscontrole als online micro-betalingstransacties. Andere toepassingen van de ACR122U zijn toegangscontrole, e-betaling, e-ticketing voor evenementen en openbaar vervoer, tolheffing en netwerkverificatie.``\autocite{ACSACR122U}

Aangezien de card reader voor de POC contactloos kaarten gaat inscannen is het handig om te weten dat de card reader volgens \textcite{ACSACR122U} onder andere met de ISO 14443-4 Compliant Card, T=CL, MIFARE® Classic Card, T=CL, ISO18092 en NFC Tags protocols kan werken.

De ARC122U card reader support ook PC/SC en CT-API, dit zijn application programming interfaces die voor de integratie van smartcard technologie in een computersysteem zorgen.

Ook zijn er drivers beschikbaar op de site van ACS. De laatste driver voor Windows dateert wel van 20 maart 2018 en er wordt duidelijk gemaakt dat de ACR122U end-of-life is, dus dat er waarschijnlijk geen nieuwe drivers meer gaan worden uitgebracht door ACS.




\section{Contactless smart cards}
\subsection{Inleiding tot de smart cards}
De badges die voor de Proof of Concept zullen worden gebruikt zijn MiFare-kaarten, dit zijn contactloze smartcards, vandaar deze informatie over (contactloze) smartcards.

Volgens \textcite{SwatiTawdeSmartCard} is een smartcard een fysieke kaart met een ingebouwde chip die dient als veiligheidstoken. Zij zijn gewoonlijk even groot als een bankkaart en kunnen gemaakt zijn van metaal of plastiek, zoals een rijbewijs of een kredietkaart. De verbinding met een card reader verloopt via direct fysiek contact of via een draadloos netwerkprotocol op korte termijn, zoals herkenning van radiofrequenties of communicatie in het nabije veld. Een microcontroller of een ingebouwde geheugenchip kan de chip op een intelligente kaart zijn. 

Smartcards zijn ontworpen om immuun te zijn voor manipulatie en gebruiken encryptie om informatie in het geheugen te beveiligen. Kaarten met een microcontroller-chip kunnen verwerkingsfuncties op de kaart uitvoeren en hebben toegang tot de gegevens in het geheugen van de chip. Smartcards worden voor verschillende doeleinden gebruikt, maar vooral voor kredietkaarten en andere betaalkaarten. De verkoop van deze intelligente kaarten is de laatste jaren gestimuleerd door een drang van de betaalkaartenindustrie om EMV-conforme smartcards te gebruiken. 

\subsection{Karakterisering van een smart card}
``Smartcards hebben verschillende gezichten, hoofdzakelijk afhankelijk van het type geïntegreerde circuitchip (ICC) dat in de plastic kaart is ingebouwd en de fysieke vorm van het verbindingsmechanisme tussen de kaart en de lezer. Het kunnen zeer goedkope tokens zijn voor financiële transacties, zoals kredietkaarten, telefoontokens of getrouwheidstokens van verschillende bedrijven. Het kunnen toegangsmunten zijn om door gesloten deuren te gaan, om in een trein te rijden of om met een auto over een tolweg te rijden. Zij kunnen fungeren als identiteitspasjes om in te loggen op een computersysteem of om toegang te krijgen tot een World Wide Web server met een geauthenticeerde identiteit. Van bijzonder belang zijn verschillende van dergelijke varianten, waaronder kaarten ...''\autocite{OreillySmartCards}

\subsection{Technologie achter de smart card}
Op de website van \textcite{STASmartCard} staat te lezen dat een smartcard communiceert via een inductietechnologie die vergelijkbaar is met die van een RFID (met datasnelheden van 106 tot 848 kbit/s). Om een transactie van data te doen moet de kaart dicht bij een antenne gehouden worden. De standaard voor een contactloze smartcard om te communiceren is ISO/IEC 14443, wat communicatie van een afstand van 10 cm toelaat. Er is ook de ISO/IEC 15693 standaard, en deze laat communicatie van een afstand tot 50 cm toe. De 2 type kaarten met deze standaard zijn type A en B. 

Een verwante contactloze technologie is RFID (radiofrequentie-identificatie). In bepaalde gevallen kan deze worden gebruikt voor toepassingen die vergelijkbaar zijn met die van contactloze smartcards, zoals voor elektronische tolheffing. RFID-apparaten hebben gewoonlijk geen schrijfbaar geheugen of microcontroller-verwerkingscapaciteit, zoals contactloze smartcards vaak wel hebben.

Een RFID-tag heeft volgens \textcite{STASmartCard} enkele voordelen zoals dat ze eenvoudig in elkaar zitten, ze zijn goedkoop en ook wegwerpbaar. Een nadeel is echter dat er weinig tot geen beveiliging op zo een RFID-tag zit, dit kan voor veiligheidsproblemen zorgen tijdens de communicatie met de lezer.

``Elke lezer die de juiste RF-frequentie (lage frequentie: 125/134 KHz; hoge frequentie: 13,56 MHz; en ultrahoge frequentie: 900MHz) en het juiste protocol gebruikt, kan de RFID-tag zijn inhoud laten meedelen. (Merk op dat dit niet geldt voor autosleutels die een beveiligde RFID-tag bevatten.) Passieve RFID-tags (d.w.z. tags zonder batterij) kunnen worden gelezen op afstanden van enkele centimeters tot vele meters, afhankelijk van de frequentie en de sterkte van het RF-veld dat voor de specifieke tag wordt gebruikt.''\autocite{STASmartCard}

\subsection{Verschillen tussen smartcard technology and RFID technology}
Volgens \textcite{STASmartCard} wordt RFID technologie eerder gebruikt om bijvoorbeeld het fabriceren en versturen van goederen te tracken over een langere afstand. Terwijl contactloze smartcards gebruik maken van RF technologie, wat over een kortere afstand werkt en beter beveiligd is.

\subsection{Voordelen van een smartcard}
Volgens \textcite{SwatiTawdeSmartCard} zijn er drie voordelen aan een smartcard. Aangezien smartcards gebruik maken van microprocessors heeft het een betere beveiliging omdat de microprocessors data direct kunnen verwerken zonder toegang vanop afstand. Smartcards hebben ook geen last van elektronische interferentie en magnetische velden. En tenslotte als de kaart geüpdate wordt moet er niet steeds opnieuw een nieuwe kaart worden geproduceerd maar wordt de data op de kaart gewoon geüpdate.

\subsection{Nadelen van een smartcard}
Een smartcard heeft zo zijn voordelen, maar volgens een artikel van \textcite{SageSmartCard} zijn er ook enkele nadelen. Een smartcard is namelijk zeer licht en kan makkelijk worden vergeten of kwijt gespeeld. De prijs van producten die gebruik maken van smartcards ligt hoog. En hoewel de beveiliging van de smartcard goed is zoals eerder vermeld bij de voordelen van een smartcard, is en blijft het nog steeds een groot risico.

\subsection{MiFare Classic kaart}
Voor de Proof of Concept gaat een MiFare card worden gebruikt, dus vandaar een beetje extra info over MiFare cards.
De MiFare card is van het merk NXP Semiconductors, op de \textcite{MiFareSmartCard} website staat te lezen dat de technologie achter de kaart is gebaseerd op de ISO/IEC 14443A standaard en werkt op een frequentie van 13.56 MHz. De MiFare card is een opslagmedium, het geheugen is verdeeld in sectoren en blokken. De kaart kost niet veel en wordt daarom ook veel gebruikt in bijvoorbeeld het openbaar vervoer, of in het geval van Jan De Nul als toegangspas.
De MiFare Classic 1K heeft een opslagruimte van 1024 Bytes en de MiFare Classic 4k beschikt over een opslagruimte van 4 kilobyte. De kaart kan worden geprogrammeerd voor lees en schrijf operaties.




\section{Uitlezen van de MiFare card tag}
\subsection{Achtergrond informatie}
Volgens \textcite{SmartcardFocus} bevat elke Mifare kaart een chip met daarin een identificatienummer, UID, dat niet kan worden veranderd. Dit permanent id wordt tijdens de fabricage al toegewezen tot een bepaalde kaart. Dit UID kan bestaan uit 4 bytes (32bit), 7 bytes (56Bit) of 10 bytes (80bit). Maar het is ook mogelijk dat een kaart een random UID genereert die niet bestaat uit een van de eerder opgesomde aantal bytes. 

\subsection{Hoe uitlezen}
Het uitlezen van de UID van een contactloze storage card kan volgens \textcite{SmartcardFocus} in de volgende drie stappen gebeuren:
\begin{itemize}
    \item Het ophalen van de context handler (de SCardEstablishContext). 
    \item Het connecteren van de kaart met de card reader (de SCardConnect). 
    \item Met behulp van de SCardTransmit kan de command worden aangeroepen die data van de kaart zal kunnen ophalen. 
\end{itemize}
Om gebruik te kunnen maken van de SCardConnect moet er in de parameters de naam van de card reader worden meegegeven. Maar om de SCardConnect te doen werken moet er een kaart op de lezer worden geplaatst. Het is ook mogelijk om te wachten tot de kaart tegen de lezer wordt gehouden, maar dit kost wat extra werk, hiervoor zal de onCardInput command moeten worden gebruikt.

\subsection{Benodigdheden binnen de code}
Dit is enkel een beschrijving van wat de code zal moeten bevatten om een UID uit te kunnen lezen, omdat deze code licht kan verschillen van de code die in de Proof of Concept zal worden gebruikt. 

Eerst en vooral zal er volgens \textcite{SmartcardFocus} een functie moeten bestaan die een lijst van alle geconnecteerde card readers kan ophalen. Eens de lijst is opgehaald kan er een bepaalde lezer worden uitgehaald die zal worden gebruikt om de smartcard uit te lezen, in de Proof of Concept weten we de naam van de lezer dus de lezer met de gewenste naam kan uit de lijst worden genomen. Als de lezer niet aanwezig is wordt er een exception gegooid.

De functie die de juiste cardreader zal ophalen kan dan vervolgens worden gebruikt in de functie die dan ook echt de UID uit de kaart zal lezen, de GetUID functie. Binnen de GetUID functie zal er verbinding worden gemaakt met de geselecteerde lezer met behulp van de SCardEstablishContext die wordt meegegeven als parameter in de SCardConnect. Eens de verbinding is gemaakt kan de bytestring van de kaart dat op de card reader ligt worden uitgelezen. Deze bytestring zal dan tenslotte nog worden omgezet naar een leesbare string. 




\section{Windows Services}
\subsection{Introductie}
Voor de Proof of Concept zal er ook gebruik worden gemaakt van Windows Services om een programma te maken dat op de achtergrond draait en telkens er een kaart tegen de card reader gehouden wordt de tag van de kaart uitgelezen. 

Windows Services, dat voorheen beter bekend stond als NT-services, zijn langlopende programma's die in hun eigen Windows-sessies worden uitgevoerd. Services kunnen volgens \textcite{DevMozService} automatisch worden gestart vanaf het moment dat de computer opgestart wordt en maken hiervoor geen gebruik van de gebruikersinterface. Zo een services zijn dus ideaal voor het gebruik op een server. Of voor als er een programma nodig is dat langdurig op de achtergrond zal moeten werken zonder dat het invloed heeft op andere applicaties of op de gebruiker die op dat moment de computer gebruikt. Zoals de Proof of Concept die op de achtergrond zal draaien zonder dat de gebruiker hier last van zal hebben. Ook kunnen services worden uitgevoerd op een account dat niet van de aangemelde gebruiker is of een standaard account op de computer. Op de Windows SDK-documentatie kan er eventueel nog meer info worden gevonden over Windows Services. 

``U kunt eenvoudig services maken door een toepassing te maken die als een service is geïnstalleerd. Stel dat u gegevens van prestatiemeteritems wilt bewaken en wilt reageren op drempelwaarden. U kunt een Windows-servicetoepassing schrijven die luistert naar de prestatiemeteritemgegevens, de toepassing implementeren en beginnen met het verzamelen en analyseren van gegevens.''\autocite{DevMozService}

Een Windows Service maken kan volgens \textcite{DevMozService} simpel worden gedaan door een Microsoft Visual Studio-project aan te maken. In dat project kunnen dan de opdrachten worden gecodeerd die worden verzonden naar de service, en er kan in het project ook worden gedefinieerd wat er moet gebeuren als de opdracht ontvangen wordt in de service. Er zijn verschillende opdrachten die de service kan uitvoeren, enkele voorbeelden hiervan, en degene die in de Proof of Concept zullen worden gebruikt, zijn het starten en stoppen van de Windows Service. Maar ook het onderbreken, het hervatten en aangepaste opdrachten kunnen worden gedefinieerd. 

\subsection{Levensduur van een Windows Service}
De levensduur van een Windows Service gaat volgens \textcite{DevMozService} langs verschillende fases. In de eerste fase moet de service worden geïnstalleerd waarna de service zal kunnen worden gebruikt. Dit gebeurt door het installatieprogramma van de Service uit te voeren. Daarna wordt de service geladen in de Service Control Manager, dit is het centrale programma op Windows waarin zich een lijst van Windows Services bevindt en waaruit services kunnen worden beheerd. 

De tweede fase bestaat uit het opstarten van de service, eens de service is opgestart zal die werken. Het starten van een service kan op drie manieren. Vanuit de Service Control Manager, vanuit de Server Explorer of door vanuit de code de ‘start’ methode aan te roepen. Wanneer de 'start’ methode wordt aangeroepen zal deze de ‘OnStart’ methode aanroepen in de service zelf en de code uitvoeren die daarin staat gedefinieerd. 

Eens de service is opgestart kan deze blijven bestaan tot de service gestopt of onderbroken wordt of tot de computer wordt afgesloten. Een Windows Service kan drie statussen hebben, running, paused of stopped. De status van de service kan ook worden weergegeven voor deze wordt uitgevoerd. Bijvoorbeeld als de service wordt onderbroken, dan zal de PausePending worden aangeroepen en vervolgens wordt de service gepauzeerd. Er kan tenslotte ook een query worden uitgevoerd om de status van de service te verkrijgen, of met behulp van de WaitForStatus kan er een actie worden uitgevoerd op het moment dat een bepaalde status zich voordoet. 

In de code van de Windows Service kunnen de OnStop, OnPause en OnContinue worden geprogrammeerd. De code die zich in die methodes bevindt zal worden uitgevoerd wanneer een van die statussen zich voordoet. Deze methodes kunnen ook worden getriggerd van uit de Services Control Manager of van uit de Server Explorer.

\subsection{Typen services}
``Er zijn twee soorten services die u in Visual Studio kunt maken met behulp van het .NET framework. Aan services die de enige service in een proces zijn, wordt het type Win32OwnProcess toegewezen. Aan services die een proces delen met een andere service, wordt het type Win32ShareProcess toegewezen. U kunt het servicetype ophalen door een query uit te voeren op de ServiceType eigenschap. 

Af en toe ziet u andere servicetypen als u een query uitvoert op bestaande services die niet zijn gemaakt in Visual Studio.''\autocite{DevMozService}

\subsection{Vereisten voor een Windows Service}
Er zijn volgens \textcite{DevMozService} twee vereisten waar rekening mee moet worden gehouden bij het maken van een service. De eerste is dat de service moet worden gemaakt in en Windows-servicetoepassingsproject of een .NET project waarmee een .exe-bestand wordt gemaakt en die de ServiceBase klasse overneemt. 

De tweede vereiste is dat projecten die gebruik maken van services installatieonderdelen nodig hebben om het project en bijhorende services goed te kunnen runnen. 

\subsection{Installeren en verwijderen van een service}
Het installeren van een service kan volgens \textcite{DevMozServiceInstall} met behulp van InstallUtil.exe of PowerShell. Voor de InstallUtil.exe kan dit met de ‘installutil <yourproject>.exe’ command. Met PowerShell is dit gelijkaardig, met het verschil dat je met PowerShell een naam aan de service kan geven, dit gebeurt met de ‘New-Service -Name "YourServiceName" -BinaryPathName <yourproject>.exe’ command. Verwijderen is eveneens gelijkaardig, voor InstallUtil.exe is dit met de ‘installutil /u <yourproject>.exe’ command en voor PowerShell wordt hiervoor de ‘Remove-Service -Name "YourServiceName"’ command gebruikt. 





\section{WebUSB}
\subsection{Concepten en gebruik}
Een manier om USB-apparaten te connecteren met een website in de browser is WebUSB. Deze manier heb ik ook getest, hoe dit wordt gedaan en de resultaten ervan zijn terug te vinden in het hoofdstuk waar WebUSB zal worden uitgetest.

``USB is de de-facto standaard voor bedrade randapparatuur. De USB-apparaten die u op uw computer aansluit zijn meestal gegroepeerd in een aantal apparaat klassen, zoals toetsenborden, muizen, videoapparaten, enzovoort. Deze worden ondersteund door het stuurprogramma van de klasse van het besturingssysteem. Veel van deze apparaten zijn ook toegankelijk via de WebHID API.''\autocite{DevMozWebUSB}

Volgens \textcite{DevMozWebUSB} bestaat er echter ook een groot aantal apparaten die niet thuishoren binnen een bepaalde apparaat klasse. Om deze apparaten toch te kunnen gebruiken zijn drivers nodig, maar die zijn niet toegankelijk via het web omdat het web geen gebruik maakt van de native code die nodig is om het apparaat te gebruiken. Dus als het apparaat toch gebruikt dient te worden op het web dan moeten er drivers geïnstalleerd van de fabrikant, dit is echter onhandig omdat bij het gebruik van een nieuw toestel opnieuw de drivers dienen geïnstalleerd te worden. 
Hiervoor is WebUSB een oplossing. Met WebUSB kunnen niet-gestandaardiseerde diensten voor USB-apparaten worden blootgesteld aan het web. Dus hardware fabrikanten zullen niet meer hun eigen API moeten aanbieden om de gebruikers van hun apparaat de mogelijkheid te bieden om het te gebruiken op het web. 
Bij het gebruik van WebUSB geeft de browser, telkens dat er een WebUSB-compatibel apparaat wordt aangesloten, een melding met daarin de link naar de website van de fabrikant. Bij het verbinden met de site gaat er een pop up verschijnen die een lijst met alle USB-apparaten weergeeft. Uit die lijst kan dan het gewenste USB-apparaat gekozen worden om de website toegang te geven tot het gebruik van dat apparaat. Hiervoor moeten geen drivers geïnstalleerd worden. 

\subsection{WebUSB interfaces}
Volgens \textcite{DevMozWebUSB} bevat WebUSB een aantal interfaces waar de user gebruik van kan maken om de verbinding met een USB-apparaat te programmeren. Bijvoorbeeld met USBConnectionEvent kan er aan het connecten of disconnecten van een USB-apparaat een event worden gekoppeld. Of USBConfiguration bevat informatie over de configuratie van het USB-apparaat en de interfaces die de kaart support.
Om dan ook echt toegang te krijgen tot een verbinden met een USB-apparaat zal er gebruik moeten worden gemaakt van de USB.requestDevice() of USB.getDevices() functies. Bij de USB.requestDevice() functie kan ook de vendorId worden meegegeven zodat alleen gewenste USB-apparaten met deze id tevoorschijn zullen komen. Voor de rest kan er met beide functies de productnaam en de fabrikantennaam worden weergegeven.




\section{Server-sent events}
\subsection{Introductie tot server-sent events}
Voor de Proof of Concept van deze bachelorproef zal gebruik worden gemaakt van server-sent events, dat verwerkt zit in het SignalR framework waar later nog meer uitleg over komt. Maar wat zijn deze server-sent events nu eigenlijk?

Volgens \textcite{DevMozSSE} is het maken van een webapplicatie die gebruik maakt van server-sent events zeer eenvoudig. Het begint allemaal met code in de server die de gewenste data naar de frontend streamt. In de frontend moet er dan code worden voorzien die de data verzonden vanuit de server kan opvangen. Deze code lijkt op het afhandelen van websockets. Deze verbinding werkt met eenrichtingsverkeer, er kan enkel data verzonden worden van de server naar de client maar niet van de client naar de server. 

\subsection{EventSource}
Om in de frontend een connectie te openen met de server zal er volgens \textcite{DevMozSSE} een EventSource instantie moeten worden aangemaakt. Deze EventSource zal dan telkens de server messages verstuurd richting de frontend deze message ontvangen, waardoor de message vervolgens gebruikt zal kunnen worden in de frontend.
Als de message van de server geen event field heeft, zal deze worden ontvangen als 'message events' maar als de message van de server wel een event field bevat worden deze ontvangen als events met de naam dat er bij wordt meegegeven. Errors zullen dan op hun beurt worden doorgegeven als error events, deze kunnen ook geprogrammeerd worden door gebruik te maken van de onerror callback van het EventSource object. De event stream kan tenslotte ook worden gesloten met behulp van de .close() method van het EventSource object.

\subsection{Event stream format}
``De Event stream is een eenvoudige stroom van tekstgegevens die met UTF-8 moet worden gecodeerd. Berichten in de gebeurtenisstroom worden gescheiden door een paar newline-tekens. Een dubbele punt als eerste karakter van een regel is in wezen commentaar, en wordt genegeerd.''\autocite{DevMozSSE}

Een bericht verzonden met server-sent events bestaat volgens \textcite{DevMozSSE} uit één of meerdere tekstregels met velden voor het bericht. Elk veld binnen het bericht wordt voorgesteld door een veldnaam, gevolgd door een dubbele punt en tenslotte de tekstgegevens voor de waarde van het veld.

Om ervoor te zorgen dat de verbinding niet wegvalt wordt er gebruik gemaakt van commentaarregels. De server zal dan om de zoveel tijd een commentaarregel versturen om de verbinding in leven te houden. 

\subsection{Message fields}
De message die wordt ontvangen kan volgens \textcite{DevMozSSE} bestaan uit de volgende fields: Event, data, id en retry.
\begin{itemize}
    \item Event field: Deze zal een string bevatten die het type event zal omschrijven. Als deze string meegegeven wordt zal er een event worden uitgestuurd op de browser naar de listener die het event met die specifieke naam kan ontvangen. Om te kunnen listenen voor events zal de frontend een addEventListener() moeten bevatten. Stel dat er geen string wordt meegegeven die het event type omschrijft dan zal de ‘onmessage’ handler worden aangeroepen.
    \item Data field: Als de EventSource meerdere data lines ontvangt zal het die lines concateneren en er telkens een newline character tussen plaatsen.
    \item Id field: Dit field zal een event ID bevatten dat kan worden gebruikt om het EventSource’s laatste event ID aan te duiden.
    \item Retry field: Stel dat de connectie met de server verbroken zou worden, dan geeft het retry field aan hoe lang de browser moet wachten voor het zal proberen te reconnecten. Dit field kan enkel een Integer bevatten.
\end{itemize}




\section{SignalR}
\subsection{Wat is SignalR}
SignalR zorgt volgens de tutorial van \textcite{PluralsightSignalR} voor real-time functionaliteit, werken met client types, het sturen van berichten naar groepen en voor authenticatie en autorisatie. Bijvoorbeeld, als de ene gebruiker op de website een value aanpast, die value ook in real-time wordt aangepast voor andere gebruikers. 

\subsection{Hoe werkt SignalR}
SignalR werkt met een Hub en Clients. In de tutorials op \textcite{PluralsightSignalR} wordt uitgelegd dat een Hub een class is die de connectie met de Clients zal onderhouden, elke client is de browser die de gebruiker van de webapplicatie gebruikt. De connectie is langs beide kanten, dus wanneer de connectie tussen de Hub en de Client is gevormd kan de Hub met de Client communiceren, alsook in de omgekeerde richting. Als een Client een message verstuurt naar de Hub, bijvoorbeeld een verandering van een value, dan kan de Hub ervoor zorgen dat de verandering bij alle Clients wordt weergegeven. Een Hub kan in alle soorten applicaties worden gebruikt, zo lang het maar een server side application is. 

SignalR maakt gebruik van een Hub protocol, dat het format omschrijft die de messages moeten hebben. Normaal gezien gebruikt het Hub protocol Json als message format.

\subsection{Transports}
Er zijn volgens \textcite{PluralsightSignalR} drie manieren om met SignalR messages te versturen, websockets, server-sent events en long polling. Websockets is de beste van de drie, dit is omdat het een langdurige connectie is, die 2-way en full-duplex werkt. De andere twee transports simuleren een 2-way connection door normale HTTP requests te gebruiken. Server-sent events kunnen server to client HTTP requests uitvoeren. En tenslotte long polling, waarbij de client een request stuurt naar de server en de server pas gaat terugsturen als het iets heeft om terug te sturen. Long polling verbruikt dus nog meer resources dan server-sent events. Wanneer websockets niet beschikbaar zijn gaat het overschakelen naar server-sent events, en als deze niet beschikbaar zijn gaat het overschakelen naar long polling.
Voor de Proof of Concept zullen vooral server-sent events belangrijk zijn. 