%---------- Inleiding ---------------------------------------------------------

\section{Introductie}%
\label{sec:introductie}

Voor het inlezen van MiFare kaarten worden card readers gebruikt. Deze worden deze verbonden met een web app die op de computer zelf draait. Echter, in deze bachelorproef gaat er onderzoek worden gedaan naar een driver/extensie die het mogelijk maakt om de card reader te verbinden met een web applicatie. Op deze manier kunnen bijvoorbeeld toegangspassen worden ingelezen. Dit is handig om te weten wie er zich in het gebouw bevindt, voor het geval het gebouw bijvoorbeeld geëvacueerd moet worden.

%---------- Stand van zaken ---------------------------------------------------

\section{State-of-the-art}%9
\label{sec:state-of-the-art}

\subsection{Wat is een browser extensie}
Op \textcite{Desktop.com} is te lezen dat een browser extensie een klein software programmatje is dat gebruikt kan worden om je web browser aan te passen en te verbeteren. Daarnaast kan een extensie bestaande functies van de browser aanpassen of nieuwe functies toevoegen. Bovendien kan het de manier waarop de browser eruitziet aanpassen.
Een extensie kan bijvoorbeeld worden gebruikt om advertenties te blokkeren. Daarnaast kan het worden gebruikt om video's van het internet te downloaden.


\subsection{Wat is een driver}
Volgens \textcite{Webopedia} zorgen drivers voor de verbinding tussen een operating system en een hardware device of software applicatie. Zonder de drivers zou de hardware en software niet (goed) werken. Bovendien vertellen drivers aan de hardware of applicatie hoe ze moeten functioneren. Dit gebeurt aan de hand van requests. Enkele apparaten die gebruik maken van drivers zijn onder andere printers, controllers, modems en \textbf{card readers}.

\subsection{Wat is een MiFare kaart}
MiFare is de naam die de fabrikant aan deze kaart heeft gegeven. Volgens \textcite{Digitalid} zijn MiFare kaarten contactloze kaarten die vroeger vooral populair waren als transport passen. Echter, tegenwoordig hebben ze hun populariteit vooral te danken aan het feit dat er data op kan worden bewaard, dankzij de technologische mogelijkheden van de kaart.
Net als andere contactloze kaarten hebben de MiFare kaarten een chip. Wanneer deze chip in het magnetisch veld van een card reader komen, zal deze erop reageren. Bovendien maken ze gebruik van de ISO14443A industry-standard en werken ze op een frequentie van 13.56MHz.

\subsection{Voordelen van MiFare kaart}
Een MiFare kaart heeft volgens \textcite{Printplast} ook enkele voordelen. Ten eerste kan de kaart van een maximum afstand van 10cm worden ingelezen, wat de gebruiker een touch-and-go beleving geeft.
Daarnaast maakt het gebruik van een beveiligingsencryptie die moeilijk te clonen is. Bovendien ondersteunen MiFare kaarten een multi interface. Dit betekent dat naast contactloos inlezen ook de mogelijkheid bestaat om de kaart met contact in te lezen. Tot slot kan de technologie niet alleen worden toegepast op kaarten, maar ook bijvoorbeeld op sleutelhangers en smartphones.

\subsection{Waarvoor worden MiFare kaarten gebruikt}
Door de vele voordelen van een MiFare kaart kan deze volgens \textcite{Digitalid} voor veel verschillende doeleinden worden gebruikt. Hier volgen enkele voorbeelden: Campus- en studentenkaarten, transporttickets, eventtickets, bibliotheekkaarten, hotelsleutelkaarten en bankkaarten.

\subsection{Wat zijn Web NFC en Web USB}
Volgens een artikel van \textcite{FrançoisBeaufortUSB} zorgt Web USB ervoor dat USB-apparaatservices kunnen worden gebruikt op het web. Met deze API hebben hardwarefabrikanten de mogelijkheid om cross-platform SDK's te bouwen voor hun apparaten. Het belangrijkste voordeel hiervan is dat het USB veiliger maakt door het naar het web te brengen.

NFC staat voor "Near Field Communication". Volgens \textcite{FrançoisBeaufortNFC} vindt de communicatie tussen het scanapparaat en de tag plaats op een afstand van minder dan 10 centimeter. Met behulp van Web NFC kan men via websites lees- en schrijfacties uitvoeren op een NFC-tag die zich in de buurt van het scanapparaat bevindt. Web NFC maakt gebruik van het NFC Data Exchange Format (NDEF) voor gegevensuitwisseling omdat de beveiligingseigenschappen van NDEF gemakkelijker kwantificeerbaar zijn.

Over het algemeen geldt dat als de kaartlezer een USB-interface gebruikt en nauwkeurige controle over de communicatie met het apparaat nodig is, Web USB waarschijnlijk de betere optie is. Aan de andere kant, als de kaartlezer NFC gebruikt en een geavanceerde API vereist is voor het lezen en schrijven van NFC-tags, is Web NFC wellicht de betere optie.

\subsection{Stappen om de driver te maken}
Om de driver te maken die ervoor moet zorgen dat de kaartlezer met de browser kan worden verbonden kunnen over het algemeen de volgende stappen worden gevolgd:
\begin{itemize}
    \item 1. Bepaal de communicatie-interface van de kaartlezer (bijv. USB, seriële poort, Bluetooth).
    \item 2. Schrijf de code voor de communicatie tussen de kaartlezer en de computer via de interface.
    \item 3. Implementeer de code voor de communicatie-interface als onderdeel van het besturingssysteem van de computer.
    \item 4. Schrijf de code om de kaartlezer te detecteren en een bericht te sturen naar de webbrowser.
    \item 5. Implementeer de code in de webbrowser om de kaartlezer te detecteren en met de juiste toestemming toegang te krijgen tot de kaartinformatie.
\end{itemize}

\subsection{Server-Sent Events}
Server-sent Events zorgen, volgens \textcite{DigitalOceanSSE}, voor real-time updates van data in een webapplicatie. Aan de clientkant wordt hiervoor de EventSource API gebruikt. Door middel van deze API kan er een verbinding worden gemaakt met de server en kunnen updates worden ontvangen. Dit betekent dat wanneer de server een update verstuurt, de client direct deze data van de server ontvangt en de website direct kan worden bijgewerkt.

\subsection{Windows Services}
Windows services zijn volgens \textcite{MicrosoftWS} applicaties die voor lange tijd blijven draaien. Ze kunnen automatisch worden opgestart wanneer de computer wordt gestart en hebben geen gebruikersinterface. Hierdoor draaien services vaak op de achtergrond zonder dat de gebruiker er iets van merkt. Via de Windows Service Manager kunnen services worden beheerd. In het geval van het uitlezen van een MiFare kaart en het doorsturen van de data ervan, is een service ideaal. Door de kaart op de card reader te plaatsen, kan de service de gegevens van die kaart verwerken voor de gebruiker.

%---------- Methodologie ------------------------------------------------------
\section{Methodologie}%
\label{sec:methodologie}

In de eerste fase van het onderzoek wordt er een literatuurstudie gedaan over het maken van een driver, MiFare- en smartcards en de card reader. Als mogelijk alternatief wordt er ook gekeken naar Windows Services en SignalR. Vervolgens wordt elke potentiële oplossingen indien mogelijk uitgewerkt tot een Proof of Concept. Deze Proof of Concepts zullen de tag van de smartcard moeten kunnen uitlezen en ze op de webapplicatie doen verschijnen in een input veld. En tenslotte wordt er een vergelijkende studie gedaan op basis van de Proof of Concepts. Om te kijken welke mogelijke oplossing het beste is. Bij het vergelijken wordt er ten eerste gekeken of de Proof of Concept daadwerkelijk werkt en ten tweede wordt de tag binnen een seconde op de webapplicatie getoond. De uiteindelijke PoC wordt door Jan De Nul HR gebruikt om MiFare cards in te lezen op hun webapplicatie. 

\begin{center}
    \includegraphics[width=4cm]{flowChart4}
\end{center}

%---------- Verwachte resultaten ----------------------------------------------
\section{Verwacht resultaat}%
\label{sec:verwachte_resultaten}

Het verwachte resultaat is een werkende extensie/driver die ervoor zorgt dat een webapplicatie MiFare kaarten rechtstreeks in kan lezen in de browser. Vervolgens worden de toegangskaarten of maaltijdcheques via die webapplicatie ingescand om zo het start/stop uur in te geven. Of om gebruik te kunnen maken van de cheques door de persoon die gelinkt is aan de kaart.

\section{Verwachte conclusies}%
\label{sec:Verwachte_conclusies}

Uit deze bachelorproef moet duidelijk blijken dat het mogelijk is om een card reader rechtstreeks te verbinden met een web app vanuit de browser. De PoC zal aantonen dat dit mogelijk is en dat er dus MiFare kaarten ingescand kunnen worden in de web app. Waardoor die gebruik zal kunnen maken van data op de kaart om toegangskaarten of maaltijdcheques te kunnen beheren. Hopelijk vloeit uit dit onderzoek een PoC die klaar is voor de consument die opzoek is naar een extensie/driver voor het inlezen van chipkaarten in zijn webapplicatie.